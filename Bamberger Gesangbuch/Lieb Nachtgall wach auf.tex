\ifx\mxversion\undefined
  \input musixtex
  \input musixadd
  \input musixlyr
  \input musixsty
  \input songs
\fi
%\smallmusicsize
\normalmusicsize

%
% Version 0.1 2012-09-16
%

\fulltitle{Lieb Nachtigall, wach auf}
 \author{Bamberger Gesangbuch}
 \shortauthor{Bamberger Gesangbuch}
 \title{Lieb Nachtigall, wach auf}
 \def\voices{2}
 \maketitle

\instrumentnumber{2}
\interstaff{10}
\generalmeter{\meterfrac{2}{4}}
\generalsignature{1}

\overfullrule=0pt

\grouptop{1}{2}
\groupbottom{1}{1}
\sepbarrules
\nobarnumbers
\stafftopmarg=7\Interligne
\interinstrument=1\Interligne
\startmuflex

%\setsongraise{2}{-2mm}
\setlyrics{lyrics1}{\lyrlayout{\lyrfont}Lieb Nach-ti-gall, wach auf, wach auf, du sch\"o-nes V\"o-gel-ein auf je-nem gr\"u-nem Zwei-ge-lein, wach hur-tig ohn Ver-schnauf!
Dem Kin-de-lein sing, sing, sing dem zar-ten Je-su-lein, dem Je-su-lein!}
\setlyrics{lyrics2}{\lyrlayout{\lyrfont}Lieb Nach-ti-gall, wach auf, wach auf, du sch\"o-nes V\"o-gel-ein auf je-nem gr\"u-nem Zwei-ge-lein, wach hur-tig ohn Ver-schnauf!
Dem Kin-de-lein aus-er-ko-ren, heut ge-bo-ren, halb er-fro-ren, sing, sing, sing, sing dem zar-ten Je-su-lein!}
\assignlyrics{1}{lyrics1}
\assignlyrics{2}{lyrics2}
\startpiece
\Notes\sk\ds\nextinstrument\sk\ca{d}\en%
\bar% 1
\Notes\qp\sk\ds\ca{d}\nextinstrument\ca{gghh}\en%
\bar% 2
\Notes\ca{gghh}\nextinstrument\qa{i}\ds\ca{d}\en%
\bar% 3
\Notes\qa{i}\sk\ds\ca{d}\nextinstrument\ca{gghh}\en%
\bar% 4
\Notes\ca{gghh}\nextinstrument\ca{iihd}\en%
\bar% 5
\Notes\ca{iihd}\nextinstrument\ca{gghh}\en%
\alaligne% 6
\Notes\ca{ggh}\nextinstrument\ca{iih}\en%
\notes\ca{h}\nextinstrument\beginmel\ibbl{1}{k}{-1}\qb{1}{k}\endmel\tbl{1}\qb{1}{j}\en%
\bar% 7
\Notes\ca{kkjj}\nextinstrument\ca{iihh}\en%
\bar% 8
\Notes\ca{iihh}\nextinstrument\qa{g}\sk\ds\ca{i}\en%
\bar% 9
\Notes\qa{g}\sk\ds\ca{g}\nextinstrument\qa{i}\sk\qa{h}\en%
\generalmeter{\meterfrac{3}{4}}%
\changecontext% 10
\NOtes\qa{gf}\nextinstrument\ha{h}\en%
\Notes\qa{g}\nextinstrument\ca{ii}\en%
\generalmeter{\meterfrac{2}{4}}%
\alaligne% 11
\Notes\ha{f}\nextinstrument\ca{kkhh}\en%
\bar% 12
\Notes\ha{e}\nextinstrument\ca{jjgg}\en%
\generalmeter{\meterfrac{4}{4}}%
\changecontext% 13
\Notes\qup{d}\sk\sk\ca{f}\beginmel\ibu{0}{g}{-1}\qb{0}{g}\endmel\tbu{0}\qb{0}{f}\beginmel\ibu{0}{h}{-1}\qb{0}{h}\endmel\tbu{0}\qb{0}{g}%
\nextinstrument\ca{ii}\qa{h}\sk\qa{i}\sk\qa{j}\en%
\bar% 14
\Notes\qa{f}\sk\qa{f}\sk\qa{g}\sk\beginmel\ibu{0}{h}{1}\qb{0}{h}\endmel\tbu{0}\qb{0}{g}%
\nextinstrument\qlp{k}\sk\sk\ca{j}\beginmel\ibu{1}{j}{-1}\qb{1}{j}\endmel\tbu{1}\qb{1}{i}\beginmel\ibu{1}{h}{-1}\qb{1}{h}\endmel\tbu{1}\qb{1}{g}\en%
\bar% 15
\NOtesp\qlp{j}\nextinstrument\qup{h}\en%
\Notes\ca{i}\nextinstrument\ca{g}\en%
\NOtesp\qlp{i}\nextinstrument\qup{g}\en%
\Endpiece

\vskip2\baselineskip
{\lyrfont
\vskip\baselineskip
\halign{\vtop{\hsize=3em\hfill #}\tabskip=1em&\vtop{\hsize=16cm #}\cr
2.&Flieg her zum Krippelein! Flieg her, gefiedert Schwesterlein, blas an dem feinen Psalterlein, sing, Nachtigall, gar fein!
Dem Kindelein musiziere, koloriere, jubiliere. Sing, sing, sing sing dem s\"u\ss en Jesulein!\cr
\cr
3.&Stimm, Nachtigall, stimm an! Den Takt gib mir dem Federlein, auch freudig schwing die Fl\"ugelein, erstreck dein H\"alselein!
Der Sch\"opfer dein Mensch will werden mit Geb\"arden hier auf Erden: Sing, sing, sing, sing dem werten Jesulein!\cr
}
\vfill
\hrule
\vskip\baselineskip
Bamberger Gesangbuch 1670

\bye