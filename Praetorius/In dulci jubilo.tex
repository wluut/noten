\ifx\mxversion\undefined
  \input musixtex
  \input musixadd
  \input musixlyr
  \input musixsty
  \input songs
\fi
%\smallmusicsize
\normalmusicsize

%
% Version 1.0 2013-11-17
%

\fulltitle{In dulci jubilo}
 \author{Michael Pr\"atorius}
 \shortauthor{Michael Pr\"atorius}
 \title{In dulci jubilo}
 \def\voices{SATB}
 \maketitle

\instrumentnumber{2}
\interstaff{18}
\generalmeter{\meterfrac{6}{4}}
\generalsignature{1}
\setclef{1}{\bass}

\elemskip=1.4\elemskip
\overfullrule=0pt

\grouptop{1}{2}
\groupbottom{1}{1}
\sepbarrules
\nobarnumbers
\stafftopmarg=7\Interligne
\setinterinstrument{1}{4\Interligne}
\startmuflex

\setsongraise{2}{1ex}
\setlyrics{lyricsa}{\lyrlayout{\lyrfont}In dul-ci ju-bi-lo, nun sin-get und seid froh! Uns-res Her-zens Won-ne leit in pr\ae-se-pi-o und leuch-tet als die Son-ne ma-tris in gre-mi-o.
Al-pha es et O, Al-pha es et O.}
\setlyrics{lyricsb}{\lyrlayout{\lyrfont\it}Nun sin-get und seid froh, jauchzt al-le und sagt so: Un-sers Her-zens Won-ne liegt in der Krip-pen blo\ss{} und leucht' doch wie die Son-ne
in sei-ner Mut-ter Scho\ss. Du bist A und O, du bist A und O.}
\assignlyrics{2}{lyricsa,lyricsb}

\startpiece
\Notes\zql{G}\qu{N}%
\nextinstrument\zql{d}\qu{g}\en%
\bar% 1
\Notes\zhl{G}\hu{b}\sk\zql{G}\qu{b}%
\nextinstrument\zhl{d}\hu{g}\sk\zql{d}\qu{g}\en%
\notes\ibsluru{1}{b}\ibu{1}{b}{1}\zhl{N}\qb{1}{b}\tbu{1}\qb{1}{c}\tbsluru{1}{d}\qu{d}\sk\zql{L}\qu{e}\sk%
\nextinstrument\beginmel\ibslurd{2}{d}\zhu{i}\ql{d}\sk\endmel\tbslurd{2}{g}\ql{g}\sk\zql{g}\qu{j}\sk\en%
\bar% 2
\notes\ibslurd{0}{K}\ibsluru{1}{f}\zhl{K}\qu{f}\sk\ibu{1}{e}{-1}\qb{1}{e}\tbu{1}\qb{1}{d}\zql{a}\qu{^c}\sk\tbslurd{0}{K}\tbsluru{1}{d}\zhl{K}\hu{d}\sk\sk\sk\zql{K}\qu{a}\sk%
\nextinstrument\beginmel\ibslurd{2}{h}\ibsluru{3}{k}\zhu{k}\hlp{h}\sk\sk\sk\qu{l}\sk\endmel\tbslurd{2}{f}\tbsluru{3}{k}\zhl{f}\hu{k}\sk\sk\sk\zql{f}\qu{k}\sk\en%
\bar% 3
\notes\ibslurd{0}{L}\ibsluru{1}{e}\ibl{0}{L}{1}\zqu{e}\qb{0}{L}\tbl{0}\qb{0}{M}%
\nextinstrument\zhl{b}\hu{g}\en%
\Notes\tbslurd{0}{N}\tbsluru{1}{d}\zql{N}\qu{d}\zql{G}\qu{b}\zhl{N}\hu{b}\sk\zql{a}\qu{e}%
\nextinstrument\sk\zql{d}\qu{g}\zhl{d}\hu{i}\sk\zql{h}\qu{j}\en%
\bar% 4
\Notes\ibslurd{0}{b}\ibsluru{1}{d}\zhl{b}\hu{d}\sk\zql{a}\qu{^c}\tbslurd{0}{K}\tbsluru{1}{d}\pt{K}\zhl{K}\hup{d}\sk\sk%
\nextinstrument\beginmel\ibslurd{2}{f}\ibsluru{3}{k}\zhl{f}\hu{k}\sk\zql{h}\qu{l}\endmel\tbslurd{2}{h}\tbsluru{3}{k}\pt{h}\zhl{h}\hup{k}\sk\sk\en%
\bar% 5
\Notes\zhl{N}\hu{b}\sk\zql{J}\qu{c}\zhl{G}\hu{b}\sk\zql{H}\qu{a}%
\nextinstrument\zhl{g}\hu{k}\sk\zql{g}\qu{l}\zhl{g}\hu{k}\sk\zql{e}\qu{j}\en%
\bar% 6
\Notes\pt{L}\zhl{L}\hup{N}\sk\sk\dotted\ibslurd{0}{L}\dotted\ibsluru{1}{b}\zhl{L}\hu{b}\sk\tbslurd{0}{L}\tbsluru{1}{b}\zql{L}\qu{b}%
\nextinstrument\pt{e}\zhl{e}\hup{i}\sk\sk\dotted\ibslurd{0}{e}\dotted\ibsluru{1}{g}\zhl{e}\hu{g}\sk\tbslurd{0}{e}\tbsluru{1}{g}\zql{e}\qu{g}\en%
\bar% 7
\Notes\zhl{K}\hu{d}\sk\zql{K}\qu{d}\zhl{N}\hu{d}\sk\zql{K}\qu{d}%
\nextinstrument\zhl{f}\hu{h}\sk\zql{f}\qu{h}\zhl{g}\hu{i}\sk\zql{f}\qu{h}\en%
\bar% 8
\Notesp\ibslurd{0}{L}\ibsluru{1}{d}\zhu{b}\qlp{L}%
\nextinstrument\beginmel\ibslurd{2}{e}\ibsluru{3}{g}\zhu{g}\hlp{e}\en%
\notes\cl{K}\en%
\Notes\zql{J}\qu{a}\tbslurd{0}{I}\tbsluru{1}{M}\zhl{I}\hu{M}\sk\zql{L}\qu{N}%
\nextinstrument\qu{h}\endmel\tbslurd{2}{d}\tbsluru{3}{i}\zhl{d}\hu{i}\sk\zql{e}\qu{i}\en%
\bar% 9
\Notes\zhl{I}\hu{b}\sk\zql{J}\qu{N}\zhl{G}\hu{b}\sk\zql{H}\qu{a}%
\nextinstrument\zhl{f}\hu{k}\sk\zql{e}\qu{l}\zhl{g}\hu{k}\sk\zql{e}\qu{j}\en%
\bar% 10
\Notes\pt{L}\zhl{L}\hup{N}\sk\sk\zhl{L}\hu{b}\sk\zql{L}\qu{b}%
\nextinstrument\pt{e}\zhl{e}\hup{i}\sk\sk\zhl{e}\hu{g}\sk\zql{e}\qu{g}\en%
\bar% 11
\Notes\zhl{H}\hu{^c}\sk\zql{K}\qu{d}\zhl{G}\hu{d}\sk\zql{K}\qu{d}%
\nextinstrument\zhl{e}\hu{h}\sk\zql{f}\qu{h}\zhl{g}\hu{i}\sk\zql{f}\qu{h}\en%
\bar% 12
\notes\ibslurd{0}{L}\ibsluru{1}{b}\ibl{0}{L}{1}\zhu{b}\qb{0}{L}\tbl{0}\qb{0}{M}%
\nextinstrument\beginmel\ibslurd{2}{e}\ibsluru{3}{g}\zql{e}\hu{g}\en%
\Notes\ql{N}\zql{K}\qu{M}\tbslurd{0}{G}\tbsluru{1}{N}\pt{G}\zhl{G}\hup{N}\sk\sk%
\nextinstrument\itenl{5}{d}\hl{d}\qu{h}\endmel\tten{5}\tbslurd{2}{d}\tbsluru{3}{i}\pt{d}\zhl{d}\hup{i}\sk\sk\en%
\bar% 13
\Notes\zhl{J}\hu{N}\sk\zql{H}\qu{a}\zhl{K}\hu{a}\sk\zql{I}\qu{b}%
\nextinstrument\zhl{c}\hu{e}\sk\zql{c}\qu{e}\zhl{d}\hu{f}\sk\zql{d}\qu{f}\en%
\bar% 14
\NOTesp\ibslurd{0}{L}\ibsluru{1}{b}\pt{L}\zhl{L}\hup{b}\tbslurd{0}{K}\tbsluru{1}{M}\pt{K}\zhl{K}\hup{M}%
\nextinstrument\beginmel\ibslurd{2}{e}\ibsluru{3}{g}\pt{e}\zhl{e}\hup{g}\endmel\tbslurd{2}{h}\tbsluru{3}{k}\pt{h}\zhl{h}\hup{k}\en%
\bar% 15
\Notes\ibslurd{0}{L}\ibsluru{1}{N}\zql{L}\qu{N}\tbslurd{0}{M}\tbsluru{1}{a}\zql{M}\qu{a}\zql{N}\qu{b}%
\ibslurd{0}{K}\ibsluru{1}{M}\zql{K}\qu{M}\tbslurd{0}{J}\tbsluru{1}{L}\zql{^J}\qu{L}\zql{K}\qu{M}%
\nextinstrument\beginmel\zql{g}\hu{i}\endmel\hl{d}\qu{i}\beginmel\ibslurd{2}{d}\zql{d}\hu{h}\endmel\tbslurd{2}{e}\ql{e}\zql{d}\qu{h}\en%
\bar% 16
\NOTesp\ibslurd{0}{G}\ibsluru{1}{N}\pt{G}\zhl{G}\hup{L}\tbslurd{0}{G}\tbsluru{1}{b}\pt{G}\zhl{G}\hup{b}%
\nextinstrument\beginmel\ibslurd{2}{d}\ibsluru{3}{g}\pt{d}\zhl{d}\hup{g}\endmel\tbslurd{2}{d}\tbsluru{3}{g}\pt{d}\zhl{d}\hup{g}\en%
%
\Endpiece

{\lyrfont
\vskip\baselineskip
\halign{\vtop{\hsize=3em\hfill #}\tabskip=1em&\vtop{\hsize=7cm #}&\vtop{\hsize=3em\hfill #}&\vtop{\hsize=7cm #}\cr
2.&O Jesu parvule,~/
nach dir ist mir so weh,~/
tr\"ost mir mein Gem\"ute,~/
o puer optime,~/
durch alle deine G\"ute,~/
o princeps glori\ae.~/
Trahe me post te,~/
trahe me post te!
&2.&Sohn Gottes in der H\"oh,~/
nach dir ist mir so weh.~/
Tr\"ost mir mein Gem\"ute,~/
o Kindlein zart un rein,~/
durch alle deine G\"ute,~/
o liebstes Jesulein,~/
Zie mich hin zu dir,~/
zie mich hin zu dir.
\cr
\cr
3.&O patris caritas~/
o nati lenitas,~/
wir w\"arn all verdorben~/
per nostra crimina,~/
so hat er uns erworben~/
c\oe lorum gaudia.~/
Eia, w\"arn wir da,~/
 eia, w\"arn wir da!
&3.&Gro\ss{} ist des Vaters Huld,~/
der Sohn tilgt unsre Schuld.~/
Wir warn all verdorben~/
durch S\"und und Eitelkeid,~/
so hat er uns erworben~/
die ewig Himmelsfreud.~/
O welch gro\ss e Gnad,~/
o welch gro\ss e Gnad!
\cr
\cr
4.&Ubi sunt gaudia?~/
\lower1.2ex\hbox to 1em{\tinynotesize\qp}Nirgend mehr denn da,~/
da die Engel singen~/
nova cantica~/
\lower1.2ex\hbox to 1em{\tinynotesize\qp}und die Schellen klingen~/
in regis curia.~/
Eia, w\"arn wir da,~/
eia, w\"arn wir da!
&4.&Wo ist der Freuden Ort?~/
\lower1.2ex\hbox to 1em{\tinynotesize\qp}Nir-gends mehr denn dort,~/
da die Engel singen~/
mit den Heilgen all~/
\lower1.2ex\hbox to 1em{\tinynotesize\qp}und die Psalmen klingen~/
im hohen Himmelssaal.~/
Eia, w\"arn wir da,~/
eia, w\"arn wir da.
\cr
}
\vfill

\hrule
\vskip\baselineskip
Text: 14. Jahrundert / Hannover 1646 und Leipzig 1545 (Strophe 3)\par
Melodie: 14. Jahrhundert, Wittenberg 1529; Satz: Michael Pr\"atorius

\bye