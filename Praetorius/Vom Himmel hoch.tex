\ifx\mxversion\undefined
  \input musixtex
  \input musixadd
  \input musixlyr
  \input musixsty
  \input songs
\fi
%\smallmusicsize
\normalmusicsize

%
% Version 0.1 2012-09-16
%

\fulltitle{Vom Himmel hoch, da komm ich her}
 \author{Michel Praetorius}
 \shortauthor{Michel Praetorius}
 \title{Vom Himmel hoch, da komm ich her}
 \def\voices{SATB}
 \maketitle

\instrumentnumber{2}
\interstaff{16}
\generalmeter{\meterC}
\generalsignature{0}
\setclef{1}{\bass}

%\elemskip=.5\elemskip
\overfullrule=0pt

\grouptop{1}{2}
\groupbottom{1}{1}
\sepbarrules
\nobarnumbers
\stafftopmarg=7\Interligne
\setinterinstrument{1}{\Interligne}
\startmuflex

%\setsongraise{2}{-2mm}
\setlyrics{lyrics}{\lyrlayout{\lyrfont}\frqq Vom Him-mel hoch, da komm ich her, ich bring euch gu-te neu-e M\"ar; der gu-ten M\"ar bring ich so viel, da-von ich singn und sa-gen will.}
\assignlyrics{1}{}
\assignlyrics{2}{lyrics}
\startpiece
\notes\zcl{J}\cu{e}%
\nextinstrument\zcl{g}\cu{j}\en%
\Notes\zql{N}\qu{d}\zql{a}\qu{c}%
\nextinstrument\zql{g}\qu{i}\zql{^f}\qu{h}\en%
\bar % 1
\Notes\zql{N}\qu{d}%
\nextinstrument\zqu{i}\qlp{g}\en%
\notes\zql{L}\qu{e}%
\nextinstrument\beginmel\qu{g}\endmel\cl{e}\en%
\Notes\zql{a}\qu{c}\zql{N}\qu{d}%
\nextinstrument\zql{e}\qu{h}\zql{g}\qu{i}\en%
\bar% 2
\Notes\zql{J}\qu{e}%
\nextinstrument\zql{g}\qu{j}\en%
\notes\ds\zcl{H}\cu{e}%
\nextinstrument\ds\zcl{h}\cu{j}\en%
\Notes\zql{J}\qu{e}\zql{L}\qu{b}%
\nextinstrument\zql{g}\qu{j}\zql{e}\qu{g}\en%
\bar% 3
\Notes\zql{I}\qu{b}\zql{J}\qu{N}\zql{G}\qu{b}\zql{K}\qu{a}%
\nextinstrument\zql{d}\qu{g}\zql{c}\qu{e}\zql{d}\qu{g}\zql{d}\qu{f}\en%
\alaligne% 4
\Notes\zql{H}\qu{a}%
\nextinstrument\zql{c}\qu{e}\en%
\notes\ds\zcl{H}\cu{L}%
\nextinstrument\ds\zcl{c}\cu{e}\en%
\Notes\zql{F}\qu{M}\zql{F}\qu{M}%
\nextinstrument\zql{c}\qu{h}\zql{c}\qu{h}\en%
\bar% 5
\Notes\zql{J}\qu{L}\zql{I}\qu{N}\zql{H}\qu{L}\zql{K}\qu{^M}%
\nextinstrument\zql{c}\qu{g}\zql{d}\qu{i}\zql{e}\qu{j}\zql{d}\qu{h}\en%
\bar% 6
\Notes\zql{G}\qu{N}%
\nextinstrument\zql{b}\qu{g}\en%
\notes\ds\zcl{H}\cu{L}%
\nextinstrument\ds\zcl{e}\cu{j}\en%
\Notes\zql{L}\qu{N}%
\nextinstrument\zql{e}\qu{i}\en%
\notes\zql{M}\qu{a}%
\nextinstrument\beginmel\ibl{2}{c}{1}\zqu{h}\qb{2}{c}\endmel\tbl{2}\qb{2}{d}\en%
\bar% 7
\Notes\pt{N}\zql{N}\qup{b}%
\nextinstrument\zql{d}\qu{g}\en%
\notes\sk\itenu{1}{a}\zcl{K}\cu{a}\tten{1}\ibl{0}{H}{1}\zcu{a}\qb{0}{H}\tbl{0}\zcu{N}\qb{0}{J}%
\nextinstrument\zql{d}\qu{g}\sk\beginmel\ibu{3}{f}{-1}\zql{c}\qb{3}{f}\endmel\tbu{3}\qb{3}{e}\en%
\Notes\zql{G}\qu{N}%
\nextinstrument\zql{b}\qu{d}\en%
\bar% 8
\Notesp\pt{J}\zql{J}\qup{L}%
\nextinstrument\pt{N}\zql{N}\qup{c}\en%
%
\Endpiece

{\lyrfont
\vskip\baselineskip
\halign{\vtop{\hsize=3em\hfill #}\tabskip=1em&\vtop{\hsize=16cm #}\cr
2.&Es ist ein Kindlein heut geborn~/ von einer Jungfrau auserkorn,~/ ein Kindelein so zart und fein,~/ das soll eu'r Freund und Wonne sein.\cr
\cr
3.&Es ist der Herr Christ unser Gott,~/ der will euch f\"uhrn aus aller Not,~/ er will eu'r Heiland selber sein,~/ von allen S\"unden machen rein.\cr
\cr
4.&Er bring euch alle Seligkeit,~/ die Gott der Vater hat bereit'.~/ da\ss{} ihr mit uns im Himmelreich~/ sollt leben nun und ewiglich.\cr
\cr
5.&So merkt nun das Zeichen recht:~/ die Krippe, Windelein so schlecht,~/ da findet ihr das Kind gelegt,~/ das alle Welt erh\"alt und tr\"agt.\flqq\cr
\cr
6.&Des la\ss t uns alle fr\"ohlich sein~/ und mit den Hirten gehn hinein,~/ zu sehn, was Gott uns hat beschert,~/ mit seinem lieben Sohn verehrt.\cr
\cr
7.&Mer auf, mein Herz, und sieh dorthin;~/ was liegt doch in dem Krippelein?~/ Wes ist das sch\"one Kindelein?~/ Es ist das liebe Jesulein.\cr
\cr
8.&Sei mir willkommen, edler Gast!~/ Den S\"under nicht verschm\"ahet hast~/ und kommst ins Elend her zu mir: wie soll ich immer danken dir?\cr
\cr
9.&Ach Herr, du Sch\"opfer aller Ding,~/ wie bist du worden so gering,~/ da\ss{} du da liegst auf d\"urrem Gras,~/ davon ein Rind und Esel a\ss!\cr
\cr
10.&Und w\"ar die Welt vielmal so weit,~/ von Edelstein und Gold bereit',~/ so w\"ar sie doch dir viel zu klein,~/ zu sein ein enges Wiegelein.\cr
\cr
11.&Der Sammet und die Seiden dein,~/ das ist grob Heu und Windelein~/ darauf du K\"onig gro\ss{} und reich~/ herprangest, als w\"ar's dein Himmelreich.\cr
\cr
12.&Das hat also gefallen dir,~/ die Wahrheit anzuzeigen mir,~/ wie aller Welt Mach, Ehr und Gut~/ vor dir nichts gilt, nichts hilft noch tut.\cr
\cr
13.&Ach mein herzliebes Jesulein,~/ mach dir ein rein sanft Bettelein,~/ zu ruhen in meins Herzens Schrein,~/ da\ss{} ich nimmer vergesse dein.\cr
\cr
14.&Davon ich allzeit fr\"ohlich sei,~/ zu springen, singen immer frei~/ das rechte Susanninne sch\"on,~/ mit Herzenslust den s\"u\ss en Ton.\cr
\cr
15.&Lob, Ehr sei Gott im h\"ochsten Thron,~/ der uns schenkt seinen ein'gen Sohn.~/ Des freuet sich der Engel Schar~/ und singet uns solch neues Jahr.\cr
}
\vfill

\hrule
\vskip\baselineskip
Text: Marin Luther 1535\hfill Melodie: Martin Luther 1539

\bye