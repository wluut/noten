\ifx\mxversion\undefined
  \input musixtex
  \input musixadd
  \input musixlyr
  \input musixsty
  \input songs
\fi

%\smallmusicsize
\normalmusicsize

%
% Version 1.0 2012-05-05 Erste Version
%
\fulltitle{Lob Gott getrost mit Singen}
 \author{Werner Lamm}
 \shortauthor{Werner Lamm}
 \title{Lob Gott getrost mit Singen}
 \def\voices{SATB}
 \maketitle

\def\freqbarno{5}
\instrumentnumber{2}
\interstaff{11}
\generalmeter{\meterfrac{4}{4}}
\generalsignature{-1}
\setclef{1}{\bass}

%\elemskip1.2\elemskip

\grouptop{1}{2}
\groupbottom{1}{1}
\barnumbers
\startbarno=0
\def\freqbarno{5}
\sepbarrules
\setinterinstrument{1}{12\Interligne}
\startmuflex

\setlyrics{vers1a}{Lob Gott ge-trost mit Sin-gen, froh-lock, du christ-lich Schar!}
\setlyrics{vers1b}{Dir soll es nicht mi\ss-lin-gen, Gott hilft dir im-mer dar.}
\setlyrics{vers1c}{Ob du gleich hier mu\ss t tra-gen viel Wi-der-w\"ar-tig-keit, sollst du doch nicht ver-za-gen; er hilft aus al-lem Leid.}
\setlyrics{vers3a}{Kann und mag auch ver-las-sen ein Mut-ter je ihr Kind}
\setlyrics{vers3b}{und al-so gar ver-sto\ss-en, da\ss{} es kein Gnad mehr find't?}
\setlyrics{vers3c}{Und ob sich's m\"ocht be-ge-ben, da\ss{} sie so gar ab-fiel: Gott schw\"ort bei sei-nem Le-ben, er dich nicht las-sen will.}
\assignlyrics{2}{vers1a,vers1b,vers3a,vers3b}

\startpiece
\NOtes\sk\zql{M}\qu{M}%
\nextinstrument\verses{\lyric*{1.},,\lyric*{3.}}\sk\zql{f}\qu{f}\en%
\bar% 1
\NOtesp\zqlp{M}\qup{M}%
\nextinstrument\zqlp{f}\qup{f}\en%
\Notes\zcl{L}\cu{L}%
\nextinstrument\zcl{e}\cu{e}\en%
\NOtes\zql{M}\qu{M}\zql{N}\qu{N}%
\nextinstrument\zql{f}\qu{f}\zql{g}\qu{g}\en%
\bar% 2
\NOTes\zhl{L}\hu{L}%
\nextinstrument\zhl{e}\hu{e}\en%
\NOtes\zql{J}\qu{J}\zql{J}\qu{c}%
\nextinstrument\zql{c}\qu{c}\zql{e}\qu{g}\en%
\bar% 3
\NOtesp\zqlp{M}\qup{c}%
\nextinstrument\zqlp{f}\qup{h}\en%
\Notes\zcl{N}\cu{b}%
\nextinstrument\zcl{e}\cu{g}\en%
\NOtes\zql{a}\qu{a}\zql{N}\qu{=b}%
\nextinstrument\zql{d}\qu{f}\zql{d}\qu{d}\en%
\bar% 4
\NOTes\zhl{J}\hu{L}%
\nextinstrument\zhl{N}\hu{c}\en%
\NOtes\qp\nextinstrument\qp\en%
\advance\barno-1%
\def\atnextline{\setinterinstrument{1}{7\Interligne}}%
\setrightrepeat\alaligne%
\assignlyrics{2}{vers1c,vers3c}%
\NOtes\zql{c}\qu{c}%
\nextinstrument\zql{j}\qu{j}\en%
\bar% 5
\NOtes\zql{b}\qu{b}\zql{a}\qu{a}\zql{N}\qu{N}\zql{M}\qu{M}%
\nextinstrument\zql{i}\qu{i}\zql{h}\qu{h}\zql{g}\qu{g}\zql{f}\qu{f}\en%
\bar% 6
\NOtes\ibslurd{0}{b}\ibsluru{1}{b}\zql{b}\qu{b}\tbslurd{0}{a}\tbsluru{1}{a}\zql{a}\qu{a}\zql{N}\qu{N}\zql{M}\qu{f}%
\nextinstrument\beginmel\ibslurd{0}{i}\ibsluru{1}{i}\zql{i}\qu{i}\endmel\tbslurd{0}{h}\tbsluru{1}{h}\zql{h}\qu{h}\zql{g}\qu{g}\zql{h}\qu{j}\en%
\bar% 7
\NOtes\zql{M}\qu{d}\zql{M}\qu{c}\zql{J}\qu{c}\zql{K}\qu{a}%
\nextinstrument\zql{f}\qu{i}\zql{f}\qu{h}\zql{e}\qu{g}\zql{d}\qu{f}\en%
\bar% 8
\NOTes\zhl{G}\hu{b}\caesura\zhl{J}\hu{J}%
\nextinstrument\zhl{d}\hu{g}\caesura\zhl{c}\hu{c}\en%
\def\atnextline{\setinterinstrument{1}{12\Interligne}}%
\alaligne%\bar% 9
\NOtes\tinynotesize\zql{F}\normalnotesize\zql{M}\qu{M}\tinynotesize\zql{G}\normalnotesize\zql{N}\qu{N}\tinynotesize\zql{H}\normalnotesize\zql{a}\qu{a}\tinynotesize\zql{I}\normalnotesize\zql{b}\qu{b}%
\nextinstrument\zql{f}\qu{f}\zql{g}\qu{g}\zql{h}\qu{h}\zql{i}\qu{i}\en%
\bar% 10
\NOTes\tinynotesize\zhl{J}\normalnotesize\zhl{c}\hu{c}\tinynotesize\zhl{G}\normalnotesize\zhl{N}\hu{N}\caesura%
\nextinstrument\zhl{j}\hu{j}\zhl{g}\hu{g}\caesura\en%
\bar% 11
\NOTes\tinynotesize\zhl{I}\normalnotesize\zhl{b}\hu{b}%
\nextinstrument\zhl{i}\hu{i}\en%
\NOtes\tinynotesize\zql{H}\normalnotesize\zql{a}\qu{a}\tinynotesize\zql{G}\normalnotesize\zql{N}\qu{N}%
\nextinstrument\zql{h}\qu{h}\zql{g}\qu{g}\en%
\bar% 12
\NOtesp\ibslurd{0}{M}\ibsluru{1}{f}\zqlp{M}\qup{f}%
\nextinstrument\beginmel\ibslurd{2}{h}\ibsluru{3}{j}\zqlp{h}\qup{j}\en%
\Notes\zcl{M}\cu{d}%
\nextinstrument\zcl{f}\cu{i}\en%
\NOtes\zql{M}\qu{c}%
\nextinstrument\zql{f}\qu{h}\en%
\Notes\ibl{0}{N}{-1}\ibu{1}{N}{-1}\zqb{0}{N}\qb{1}{N}\tbl{0}\tbu{1}\zqb{0}{M}\qb{1}{M}%
\nextinstrument\ibl{2}{g}{-1}\ibu{3}{g}{-1}\zqb{2}{g}\qb{3}{g}\tbl{2}\tbu{3}\zqb{2}{f}\qb{3}{f}\en%
\bar% 13
\NOtes\tbslurd{0}{N}\tbsluru{1}{b}\zql{N}\qu{b}\zql{J}\qu{c}%
\nextinstrument\endmel\tbslurd{2}{d}\tbsluru{3}{g}\zql{d}\qu{g}\zql{e}\qu{g}\en%
\NOTes\tinynotesize\zhl{F}\normalnotesize\zhl{M}\hu{a}%
\nextinstrument\zhl{c}\hu{f}\en%
\Endpiece

{\lyrfont
\vskip\baselineskip
\halign{\vtop{\hsize=3em\hfill #}\tabskip=1em&\vtop{\hsize=16cm #}\cr
2.&Dich hat er sich erkoren,~/
durch sein Wort auferbaut,~/
bei seinem Eid geschworen,~/
dieweil du ihn vertraut,~/
da\ss{} er deiner will pflegen~/
in aller Angst und Not,~/
dein Feinde niederlegen,~/
die schm\"ahen dich mit Spott.
\cr
4.&Darum la\ss{} dich nicht schrecken,~/
o du christgl\"aub'ge Schar!~/
Gott wird dir Hilf erwecken~/
und dein selbst nehmen war.~/
Er wird seim Volk verk\"unden~/
sehr freudenreichen Trost,~/
wie sie von ihren S\"unden~/
sollen werden erl\"ost.
\cr
5.&Es tut ihn nicht gereuen,~/
was er vorl\"angst gedeut',~/
sein Kirche zu erneuen~/
in dieser f\"ahrlichn Zeit.~/
Es wird herzlich anschauen~/
dein' Jammer und Elend,~/
dich herrlich auferbauen~/
durch Wort und Sakrament.
\cr
6.&Gott solln wir fr\"ohlich loben,~/
der sich aus gro\ss{}er Gnad~/
durch seine milden Gaben~/
uns kundgegeben hat.~/
Er wird uns auch erhalten~/
in Lieb und Einigkeit~/
und unser freundlich walten~/
hier und in Ewigkeit.
\cr
}
\vfill

\hrule
\vskip\baselineskip
Text: B\"ohmische Br\"uder 1544, Melodie: 16. Jh. \frqq Entlaubet ist uns der Walde\flqq; geistlich N\"urnberg um 1535

\bye
