\ifx\mxversion\undefined
  \input musixtex
  \input musixadd
  \input musixlyr
  \input musixsty
  \input songs
\fi
%\smallmusicsize
\normalmusicsize

%
% Version 0.1 2012-09-16
%

\fulltitle{Ihr Kinderlein, kommet}
 \author{Hilger Schallehn}
 \shortauthor{Hilger Schallehn}
 \title{Ihr Kinderlein, kommet}
 \def\voices{SATB}
 \maketitle

\instrumentnumber{2}
\interstaff{16}
\generalmeter{\meterfrac{2}{4}}
\generalsignature{0}
\setclef{1}{\bass}

%\elemskip=.5\elemskip
\overfullrule=0pt

\grouptop{1}{2}
\groupbottom{1}{1}
\sepbarrules
\nobarnumbers
\stafftopmarg=7\Interligne
\setinterinstrument{1}{7\Interligne}
\startmuflex

%\setsongraise{2}{-2mm}
\setlyrics{lyrics1}{\lyrlayout{\lyrfont}Ihr Kin-der-lein, kom-met, o kom-met doch all! Zur Krip-pe her kom-met, in Beth-le-hems Stall, und seht, was in die-ser hoch-hei-li-gen Nacht
der Va-ter im Him-mel f\"ur Freu-den uns macht.}
\setlyrics{lyrics2}{\lyrlayout{\lyrfont}O seht in der Krip-pe im n\"acht-li-chen Stall, seht hier bei des Licht-leins hell-gl\"an-zen-dem Strahl in rein-li-chen Win-deln das himm-li-sche Kind,
viel sch\"o-ner und hol-der, als En-gel es sind.}
\setlyrics{lyrics3}{\lyrlayout{\lyrfont}Da liegt es, das Kind-lein, auf Heu und auf Stroh, Ma-ri-a und Jo-seph be-trach-ten es froh, die red-li-chen Hir-ten knien be-tend da-vor,
hoch o-ben schwebt ju-belnd der En-ge-lein Chor.}
\assignlyrics{1}{}
\assignlyrics{2}{lyrics1,lyrics2,lyrics3}
\startpiece\notes\sk\en%
\znotes\nextinstrument\loffset{2.2}{\verses{\lyric*{1.},\lyric*{2.},\lyric*{3.}}}\en%
\notes\zcl{N}\cu{N}%
\nextinstrument\zcl{g}\cu{g}\en%
\bar% 1
\Notes\zql{J}\qu{N}%
\nextinstrument\zql{e}\qu{g}\en%
\notes\ibl{0}{J}{0}\ibu{1}{N}{0}\zqb{0}{J}\qb{1}{N}\tbl{0}\tbu{1}\zqb{0}{J}\qb{1}{N}%
\nextinstrument\ibl{2}{c}{2}\ibu{3}{e}{2}\zqb{2}{c}\qb{3}{e}\tbl{2}\tbu{3}\zqb{2}{e}\qb{3}{g}\en%
\bar% 2
\Notes\zql{J}\qu{c}%
\nextinstrument\zql{e}\qu{g}\en%
\notes\ibl{0}{J}{0}\ibu{1}{c}{-4}\zqb{0}{J}\qb{1}{c}\tbl{0}\tbu{1}\zqb{0}{J}\qb{1}{N}%
\nextinstrument\ibl{2}{c}{2}\ibu{3}{e}{2}\zqb{2}{c}\qb{3}{e}\tbl{2}\tbu{3}\zqb{2}{e}\qb{3}{g}\en%
\bar% 3
\Notes\zql{G}\qu{N}%
\nextinstrument\zql{d}\qu{f}\en%
\notes\ibl{0}{G}{0}\ibu{1}{N}{0}\zqb{0}{G}\qb{1}{N}\tbl{0}\tbu{1}\zqb{0}{G}\qb{1}{N}%
\nextinstrument\ibl{2}{b}{2}\ibu{3}{d}{2}\zqb{2}{b}\qb{3}{d}\tbl{2}\tbu{3}\zqb{2}{d}\qb{3}{f}\en%
\bar% 4
\Notes\zql{J}\qu{N}%
\nextinstrument\zql{c}\qu{e}\en%
\notes\ds\zcl{J}\cu{N}%
\nextinstrument\ds\zcl{e}\cu{g}\en%
\bar% 5
\Notes\zql{J}\qu{N}%
\nextinstrument\zql{e}\qu{g}\en%
\notes\ibl{0}{J}{0}\ibu{1}{N}{0}\zqb{0}{J}\qb{1}{N}\tbl{0}\tbu{1}\zqb{0}{J}\qb{1}{N}%
\nextinstrument\ibl{2}{c}{2}\ibu{3}{e}{2}\zqb{2}{c}\qb{3}{e}\tbl{2}\tbu{3}\zqb{2}{e}\qb{3}{g}\en%
\bar% 6
\Notes\zql{J}\qu{c}%
\nextinstrument\zql{e}\qu{g}\en%
\notes\ibl{0}{J}{0}\ibu{1}{c}{-4}\zqb{0}{J}\qb{1}{c}\tbl{0}\tbu{1}\zqb{0}{J}\qb{1}{N}%
\nextinstrument\ibl{2}{c}{2}\ibu{3}{e}{2}\zqb{2}{c}\qb{3}{e}\tbl{2}\tbu{3}\zqb{2}{e}\qb{3}{g}\en%
\bar% 7
\Notes\zql{G}\qu{N}%
\nextinstrument\zql{d}\qu{f}\en%
\notes\ibl{0}{G}{0}\ibu{1}{N}{0}\zqb{0}{G}\qb{1}{N}\tbl{0}\tbu{1}\zqb{0}{G}\qb{1}{N}%
\nextinstrument\ibl{2}{b}{2}\ibu{3}{d}{2}\zqb{2}{b}\qb{3}{d}\tbl{2}\tbu{3}\zqb{2}{d}\qb{3}{f}\en%
\bar% 8
\Notes\zql{J}\qu{N}%
\nextinstrument\zql{c}\qu{e}\en%
\notes\ds\zcl{J}\cu{N}%
\nextinstrument\ds\zcl{c}\cu{e}\en%
\bar% 9
\notes\zql{N}\qu{N}%
\nextinstrument\beginmel\zqu{d}\roff{\ibl{2}{c}{-2}\ibslurd{2}{c}\qb{2}{c}}\endmel\tbl{2}\tbslurd{2}{b}\qb{2}{a}\en%
\notes\ibl{0}{N}{0}\ibu{1}{N}{2}\zqb{0}{N}\qb{1}{N}\tbl{0}\tbu{1}\zqb{0}{N}\qb{1}{b}%
\nextinstrument\ibl{2}{b}{2}\ibu{3}{d}{0}\zqb{2}{b}\qb{3}{d}\tbl{2}\tbu{3}\zqb{2}{d}\qb{3}{d}\en%
\bar% 10
\notes\zql{K}\qu{a}%
\nextinstrument\beginmel\zqu{f}\ibl{2}{f}{-1}\ibslurd{2}{f}\qb{2}{f}\endmel\tbl{2}\tbslurd{2}{e}\qb{2}{e}\en%
\notes\ibl{0}{M}{0}\ibu{1}{a}{2}\zqb{0}{M}\qb{1}{a}\tbl{0}\tbu{1}\zqb{0}{N}\qb{1}{b}%
\nextinstrument\ibl{2}{d}{0}\ibu{3}{f}{0}\zqb{2}{d}\qb{3}{f}\tbl{2}\tbu{3}\zqb{2}{d}\qb{3}{f}\en%
\bar% 11
\notes\zql{a}\qu{a}%
\nextinstrument\beginmel\zqu{e}\roff{\ibl{2}{d}{-2}\ibslurd{2}{d}\qb{2}{d}}\endmel\tbl{2}\tbslurd{2}{c}\qb{2}{b}\en%
\notes\ibl{0}{a}{-1}\ibu{1}{a}{2}\zqb{0}{a}\qb{1}{a}\tbl{0}\tbu{1}\zqb{0}{N}\qb{1}{c}%
\nextinstrument\ibl{2}{c}{2}\ibu{3}{e}{0}\zqb{2}{c}\qb{3}{e}\tbl{2}\tbu{3}\zqb{2}{e}\qb{3}{e}\en%
\bar% 12
\Notesp\zqlp{M}\qup{c}%
\nextinstrument\zqlp{f}\qup{h}\en%
\notes\zcl{M}\cu{c}%
\nextinstrument\zcl{f}\cu{h}\en%
\bar% 13
\Notes\zql{M}\qu{b}%
\nextinstrument\zql{d}\qu{g}\en%
\notes\ibl{0}{L}{-1}\ibu{1}{N}{-1}\zqb{0}{L}\qb{1}{N}\tbl{0}\tbu{1}\zqb{0}{K}\qb{1}{M}%
\nextinstrument\ibl{2}{c}{-1}\ibu{3}{g}{0}\zqb{2}{c}\qb{3}{g}\tbl{2}\tbu{3}\zqb{2}{b}\qb{3}{g}\en%
\bar% 14
\notes\ibl{0}{L}{2}\ibu{1}{N}{3}\ibslurd{0}{L}\ibsluru{1}{N}\zqb{0}{L}\qb{1}{N}\tbl{0}\tbu{1}\tbslurd{0}{M}\tbsluru{1}{b}\zqb{0}{N}\qb{1}{c}%
\ibl{0}{L}{-2}\ibu{1}{c}{-3}\zqb{0}{L}\qb{1}{c}\tbl{0}\tbu{1}\zqb{0}{J}\qb{1}{N}%
\nextinstrument\beginmel\ibl{2}{c}{2}\ibslurd{2}{c}\zqu{j}\qb{2}{c}\endmel\tbl{2}\tbslurd{2}{d}\qb{2}{e}%
\ibl{2}{g}{-4}\ibu{3}{g}{-2}\zqb{2}{g}\qb{3}{g}\tbl{2}\tbu{3}\zqb{2}{c}\qb{3}{e}\en%
\bar% 15
\Notes\zql{G}\qu{N}%
\nextinstrument\zql{d}\qu{f}\en%
\notes\ibl{0}{G}{0}\ibu{1}{M}{0}\zqb{0}{G}\qb{1}{M}\tbl{0}\tbu{1}\zqb{0}{G}\qb{1}{M}%
\nextinstrument\ibl{2}{d}{-4}\ibu{3}{d}{-2}\zqb{2}{d}\qb{3}{d}\tbl{2}\tbu{3}\zqb{2}{N}\qb{3}{b}\en%
\bar% 16
\Notes\zql{J}\qu{L}%
\nextinstrument\zql{c}\qu{e}\en%
\notes\ds\nextinstrument\ds\en%
\Endpiece

{\lyrfont
\vskip\baselineskip
\halign{\vtop{\hsize=3em\hfill #}\tabskip=1em&\vtop{\hsize=16cm #}\cr
4.&O beugt wie die Hirten anbetend die Knie,~/ erhebet die H\"ande und danket wie sie;~/ stimmt freudig, ihr Kinder, --- wer wollt$^{1}$ sich nicht freun? ---~/
stimmt freudig zum Jubel der Engel mit ein.\cr
\cr
5.&$^{2}$O betet: Du liebes, du g\"ottliches Kind,~/ was leidest du alles f\"ur unsere S\"und!~/ Ach hier in der Krippe schon Armut und Not,~/
am Kreuze dort gar noch den bitteren Tod.\cr
\cr
6.&So nimm unsre Herzen zum Opfer den hin;~/ wir geben die gerne mit fr\"ohlichem Sinn.~/ Ach mache sie heilig und selig wie deins~/
und mach sie auf ewig mit deinem nur eins\cr
}
\vfill

$^{1}$ Andere Fassung: sollt\par
$^{2}$ Andere Fassung: Was geben wir Kinder, was schenken wir dir,~/ du bestes und liebstes der Kinder, daf\"ur?~/ Nichts willst du von Sch\"atzen und Reichtum der Welt,~/
ein Herz nur voll Demut allein dir gef\"allt.
\hrule
\vskip\baselineskip
Text: Christop von Schmid (1798) 1811\par Melodie: Johann Abraham Peter Schulz 1794; geistlich G\"utersloh 1832

\bye