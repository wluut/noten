\ifx\mxversion\undefined
  \input musixtex
  \input musixadd
  \input musixlyr
  \input musixsty
  \input songs
\fi
%\smallmusicsize
\normalmusicsize

%
% Version 0.1 2012-09-16
%

\fulltitle{Vom Himmel hoch, da komm ich her}
 \author{Johann Sebastian Bach}
 \shortauthor{Johann Sebastian Bach}
 \title{Vom Himmel hoch, da komm ich her}
 \def\voices{SATB}
 \maketitle

\instrumentnumber{2}
\interstaff{16}
\generalmeter{\meterC}
\generalsignature{2}
\setclef{1}{\bass}

\elemskip=.5\elemskip
\overfullrule=0pt

\grouptop{1}{2}
\groupbottom{1}{1}
\sepbarrules
\nobarnumbers
\stafftopmarg=7\Interligne
\interinstrument=1\Interligne
\startmuflex

%\setsongraise{2}{-2mm}
\setlyrics{lyrics}{\lyrlayout{\lyrfont}\frqq Vom Him-mel hoch, da komm ich her, ich bring euch gu-te neu-e M\"ar; der gu-ten M\"ar bring ich so viel, da-von ich singn und sa-gen will.}
\assignlyrics{1}{}
\assignlyrics{2}{lyrics}
\startpiece
\NOtes\zql{K}\qu{f}%
\nextinstrument\zql{h}\qu{k}\en%
\bar % 1
\Notes\zql{a}\qup{e}\sk\ql{b}\cu{d}\zql{^L}\qu{c}\sk\ibu{1}{d}{-1}\zql{M}\qb{1}{d}\tbu{1}\qb{1}{c}%
\nextinstrument\beginmel\ibl{2}{h}{-1}\zqu{j}\qb{2}{h}\endmel\tbl{2}\qb{2}{g}\zql{f}\qu{i}\sk%
\beginmel\ibu{3}{j}{-1}\zql{^g}\qb{3}{j}\endmel\tbu{3}\qb{3}{i}\beginmel\ibl{2}{f}{-1}\zqu{h}\qb{2}{f}\endmel\tbl{2}\qb{2}{e}\en%
\bar% 2
\Notes\ibl{0}{N}{-1}\ibu{1}{b}{-1}\zqb{0}{=N}\qb{1}{b}\tbl{0}\tbu{1}\zqb{0}{M}\qb{1}{a}%
\nextinstrument\zql{d}\qu{i}\en%
\NOtes\zql{L}\qu{N}\fermatadown{E}\zql{K}\qu{a}%
\nextinstrument\zql{e}\qu{j}\fermataup{o}\zql{f}\qu{k}\en%
\Notes\ibl{0}{I}{1}\zqu{b}\qb{0}{I}\tbl{0}\qb{0}{J}%
\nextinstrument\zql{f}\qu{k}\en%
\bar% 3
\Notes\ibl{0}{K}{1}\ibu{1}{b}{1}\zqb{0}{K}\qb{1}{b}\tbl{0}\tbu{1}\zqb{0}{L}\qb{1}{c}%
\ibl{0}{M}{1}\zqu{d}\qb{0}{M}\tbl{0}\qb{0}{N}%
\ibl{0}{a}{1}\ibu{1}{c}{-1}\zqb{0}{a}\qb{1}{c}\tbl{0}\tbu{1}\zqb{0}{b}\qb{1}{b}\zql{=c}\hroff{\qu{a}}%
\nextinstrument\zql{f}\qu{k}\sk\zql{f}\qu{h}\sk%
\beginmel\ibu{3}{h}{-1}\zql{e}\qb{3}{h}\endmel\tbu{3}\qb{3}{g}%
\beginmel\ibu{3}{f}{1}\zql{d}\qb{3}{f}\endmel\tbu{3}\qb{3}{g}\en%
\bar% 4
\NOtes\zql{M}\qup{=c}%
\nextinstrument\zql{d}\qu{h}\en%
\Notes\ql{N}\cu{b}%
\nextinstrument\zql{d}\qu{g}\en%
\NOtes\fermatadown{E}\zql{K}\qu{a}%
\nextinstrument\fermataup{o}\zql{d}\qu{f}\en%
\Notes\ibl{0}{b}{-1}\ibu{1}{d}{-1}\zqb{0}{b}\qb{1}{d}\tbl{0}\tbu{1}\zqb{0}{a}\qb{1}{c}%
\nextinstrument\zql{d}\qu{f}\en%
\alaligne% 5
\Notes\ibl{0}{N}{-1}\ibu{1}{b}{-1}\zqb{0}{N}\qb{1}{b}\tbl{0}\tbu{1}\zqb{0}{M}\qb{1}{a}%
\ibl{0}{L}{-1}\zqu{N}\qb{0}{L}\tbl{0}\qb{0}{K}%
\ibl{0}{J}{-1}\zqu{N}\qb{0}{J}\tbl{0}\qb{0}{I}%
\ibu{1}{a}{-1}\zql{H}\qb{1}{a}\tbu{1}\qb{1}{N}%
\nextinstrument\zql{d}\qu{i}\sk\beginmel\ibl{2}{g}{-1}\zqu{i}\qb{2}{g}\endmel\tbl{2}\qb{2}{f}%
\beginmel\ibl{2}{e}{-1}\ibu{3}{h}{1}\zqb{2}{e}\qb{3}{h}\endmel\tbl{2}\tbu{3}\zqb{2}{d}\qb{3}{i}\zql{c}\qu{j}\en%
\bar% 6
\Notes\ibu{1}{M}{-2}\zqlp{I}\qb{1}{M}\tbu{1}\qb{1}{K}\qu{d}\cl{J}%
\nextinstrument\beginmel\ibl{2}{b}{5}\ibu{3}{k}{-1}\zqb{2}{b}\qb{3}{k}\endmel\tbl{2}\tbu{3}\zqb{2}{f}\qb{3}{j}\zql{g}\qu{i}\en%
\NOtes\fermatadown{E}\zql{K}\qu{d}%
\nextinstrument\fermataup{o}\zql{f}\qu{h}\en%
\Notes\ibl{0}{K}{1}\zqu{f}\qb{0}{K}\tbl{0}\qb{0}{L}%
\nextinstrument\zql{i}\qu{k}\en%
\bar% 7
\Notes\ibu{1}{f}{-5}\zqlp{M}\qb{1}{f}\tbu{1}\qb{1}{c}\qu{d}\itenl{0}{L}\cl{L}\tten{0}\zqu{c}\cl{L}\ql{K}\ibu{1}{b}{1}\qb{1}{b}\tbu{1}\zcl{J}\qb{1}{c}%
\nextinstrument\beginmel\ibl{2}{i}{-1}\loff{\zqu{j}}\qb{2}{i}\endmel\tbl{2}\qb{2}{h}\zql{g}\qu{i}\sk\zqlp{f}\qu{h}\sk%
\beginmel\ibu{3}{i}{-1}\qb{3}{i}\endmel\tbu{3}\zcl{e}\qb{3}{h}\en%
\bar% 8
\notes\ibbl{0}{I}{1}\ibu{1}{d}{-1}\zqb{1}{d}\qb{0}{I}\tbbl{0}\qb{0}{J}%
\nextinstrument\beginmel\ibu{3}{g}{-1}\zqlp{d}\qb{3}{g}\en%
\Notes\tbl{0}\tbu{1}\zqb{0}{K}\qb{1}{b}\ibl{0}{M}{-5}\zqb{0}{a}\hroff{\ibu{1}{L}{5}\qb{1}{L}}\tbl{0}\tbu{1}\zqb{0}{H}\qb{1}{a}%
\nextinstrument\endmel\tbu{3}\qb{3}{f}\beginmel\qu{e}\endmel\cl{c}\en%
\NOtes\fermatadown{E}\zql{K}\qu{M}%
\nextinstrument\fermataup{o}\zql{a}\qu{d}\en%
\Endpiece

{\lyrfont
\vskip\baselineskip
\halign{\vtop{\hsize=3em\hfill #}\tabskip=1em&\vtop{\hsize=16cm #}\cr
2.&Es ist ein Kindlein heut geborn~/ von einer Jungfrau auserkorn,~/ ein Kindelein so zart und fein,~/ das soll eu'r Freund und Wonne sein.\cr
\cr
3.&Es ist der Herr Christ unser Gott,~/ der will euch f\"uhrn aus aller Not,~/ er will eu'r Heiland selber sein,~/ von allen S\"unden machen rein.\cr
\cr
4.&Er bring euch alle Seligkeit,~/ die Gott der Vater hat bereit'.~/ da\ss{} ihr mit uns im Himmelreich~/ sollt leben nun und ewiglich.\cr
\cr
5.&So merkt nun das Zeichen recht:~/ die Krippe, Windelein so schlecht,~/ da findet ihr das Kind gelegt,~/ das alle Welt erh\"alt und tr\"agt.\flqq\cr
\cr
6.&Des la\ss t uns alle fr\"ohlich sein~/ und mit den Hirten gehn hinein,~/ zu sehn, was Gott uns hat beschert,~/ mit seinem lieben Sohn verehrt.\cr
\cr
7.&Mer auf, mein Herz, und sieh dorthin;~/ was liegt doch in dem Krippelein?~/ Wes ist das sch\"one Kindelein?~/ Es ist das liebe Jesulein.\cr
\cr
8.&Sei mir willkommen, edler Gast!~/ Den S\"under nicht verschm\"ahet hast~/ und kommst ins Elend her zu mir: wie soll ich immer danken dir?\cr
\cr
9.&Ach Herr, du Sch\"opfer aller Ding,~/ wie bist du worden so gering,~/ da\ss{} du da liegst auf d\"urrem Gras,~/ davon ein Rind und Esel a\ss!\cr
\cr
10.&Und w\"ar die Welt vielmal so weit,~/ von Edelstein und Gold bereit',~/ so w\"ar sie doch dir viel zu klein,~/ zu sein ein enges Wiegelein.\cr
\cr
11.&Der Sammet und die Seiden dein,~/ das ist grob Heu und Windelein~/ darauf du K\"onig gro\ss{} und reich~/ herprangest, als w\"ar's dein Himmelreich.\cr
\cr
12.&Das hat also gefallen dir,~/ die Wahrheit anzuzeigen mir,~/ wie aller Welt Mach, Ehr und Gut~/ vor dir nichts gilt, nichts hilft noch tut.\cr
\cr
13.&Ach mein herzliebes Jesulein,~/ mach dir ein rein sanft Bettelein,~/ zu ruhen in meins Herzens Schrein,~/ da\ss{} ich nimmer vergesse dein.\cr
\cr
14.&Davon ich allzeit fr\"ohlich sei,~/ zu springen, singen immer frei~/ das rechte Susanninne sch\"on,~/ mit Herzenslust den s\"u\ss en Ton.\cr
\cr
15.&Lob, Ehr sei Gott im h\"ochsten Thron,~/ der uns schenkt seinen ein'gen Sohn.~/ Des freuet sich der Engel Schar~/ und singet uns solch neues Jahr.\cr
}
\vfill

\hrule
\vskip\baselineskip
Text: Marin Luther 1535\hfill Melodie: Martin Luther 1539

\bye